\begin{agradecimentos}
\begin{comment}
A inclusão desta seção de agradecimentos é opcional, portanto, sua inclusão 
fica a critério do(s) autor(es), que caso deseje(em) fazê-lo deverá(ão) 
utilizar este espaço, seguindo a formatação de \textit{espaço simples e 
fonte padrão do texto (arial ou times, tamanho 12 sem negritos, aspas ou 
itálico}.
\end{comment}

Eu, Filipe Borges, gostaria de agradecer aos meus pais (Fernando e Marlene) primeiramente, por todo o apoio e suporte que eles tem me dado durante todo o curso, sem o qual jamais iria chegar aonde cheguei.

Eu, Fagner Rodrigues, quero agradecer à minha mãe (Valdevina), por toda a dedicação, ajuda, apoio e carinho. Agradeço muito, à minha avó (Pedrelina), e fica minha dedicação a ela, em sua memória, por toda sua afetividade e amor. Também agradeço a ajuda dos meus colegas e amigos, Thiago Honorato e Laís Barreto que dedicaram tempo e compartilharam conhecimento. Agradeço também aos colegas e amigos do órgão de estudo, Benoni e Marcelo por toda ajuda que ofereceram.

Agradecemos muito a Elaine Baroni, pela sua disposição em nos ajudar na construção deste trabalho. Os constantes \textit{feedbacks} e as ajudas foram de suma importância na escrita deste trabalho.

Agradecemos também:

À nossa orientadora Prof. Msc. Elaine Venson, pela coordialidade e empenho dedicados na construção do trabalho, fundamentais para que o mesmo viesse a ser concluído.

À Prof. Dra. Rejane Maria da Costa Figueiredo, pela ajuda na condução inicial do trabalho.

Aos caros colegas: Delvan Ferreira, Eduardo Garcia, Felipe de Deus, Geison de Souza, Jaime des Sousa, Jean Michel, Mardônio Rodrigues e Roberto de Sousa ficam nosso agradecimento. O tempo que disponibilizaram a este trabalho foi essencial para o avanço do mesmo.

Ao Esp. Anderson Rodrigues Ferreira, que forneceu os insumos necessários para a construção deste trabalho.

%\textbf{Caso não deseje utilizar os agradecimentos, deixar toda este arquivo
%em branco}.
\end{agradecimentos}
