\begin{apendicesenv}

\partapendices

\chapter{Perguntas da entrevista}

\begin{flushleft}
	\textbf{Descrição do desenvolvedor}
\end{flushleft}
Objetivo(s):
\begin{itemize}
	\item Obter informações sobre o perfil dos desenvolvedores.
	\begin{itemize}
		\item Código derivado: Perfil do Desenvolvedor;
	\end{itemize}
\end{itemize}
Unidade de alocação:\newline\newline    
1- Qual é o curso que você faz?\newline
2- Quantos semestres são o curso?\newline
3- Está em qual semestre do curso?\newline
4- Qual é o nome da instituição de ensino?\newline
5- Quanto tempo de estágio?\newline
\newline
\textbf{Papel do desenvolvedor na organização.}\newline\newline
Objetivo(s):
\begin{itemize}
	\item Obter informações sobre o papel do desenvolvedor no órgão.
	\begin{itemize}
		\item Código derivado: Papel do Desenvolvedor;
	\end{itemize}
\end{itemize}
6- Como você enxerga o seu papel na instituição? (A posição dentro da instituição, as responsabilidades, e etc…)\newline
7- Você considera a sua contribuição relevante para a instituição?\newline
\newline
\textbf{Efetividade do Treinamento}\newline\newline
Objetivo(s):
\begin{itemize}
	\item Obter informações sobre a preparação dos novos desenvolvedores.
	\begin{itemize}
		\item Código derivado: Preparação;
	\end{itemize}
\end{itemize}
8- Você acha que o curso o preparou bem para as tarefas que realizou?\newline
9- Qual a sua opinião com relação a efetividade do curso?\newline
10- Como você avalia a qualidade das aplicações em que deu manutenção?\newline
\newline
\textbf{Boas práticas e padrões}\newline\newline
Objetivo(s):
\begin{itemize}
	\item Obter informações sobre as práticas realizadas no desenvolvimento.
	\begin{itemize}
		\item Código derivado: Estilo e Design de codificação;
	\end{itemize}
\end{itemize}
11- Você adota algum padrão ou alguma boa prática no desenvolvimento/manutenção das aplicações?\newline
12- Teria alguma sugestão de boa prática/padrão que realiza que poderia melhorar de alguma forma a qualidade das aplicações desenvolvidas?\newline
\newline
\textbf{Requisitos}\newline\newline
Objetivo(s):
\begin{itemize}
	\item Obter informações sobre como se dá o levantamento de requisitos nos departamentos da instituição.
	\begin{itemize}
		\item Código derivado: Elicitação de Requisitos;
	\end{itemize}
	\item Obter informações de como é armazenado e gerenciado os requisitos do sistema a ser desenvolvido.
	\begin{itemize}
		\item Código derivado: Gerenciamento de Requisitos;
	\end{itemize}
	\item Obter informações de como é englobado as mudanças de requisitos ao sistema.
	\begin{itemize}
		\item Código derivado: Gerenciamento de Requisitos;
	\end{itemize}
\end{itemize}
13- Como você faz para levantar os requisitos do sistema que foi solicitado por alguma unidade?\newline
14- Como você aborda o cliente para levantar os requisitos do sistema?\newline
15- Como você armazena ou documenta o que foi passado pelo cliente (requisitos)? Existe algum meio de armazenar eles diretamente no sistema?\newline
16- Você costuma fazer reunião com o cliente em outras etapas do desenvolvimento, para levantar outros requisitos e/ou esclarecer os que já existem?\newline\newline\newline\newline
\textbf{Design}\newline\newline
Objetivo(s):
\begin{itemize}
	\item Obter informações sobre como é realizado a modelagem do sistema.
	\begin{itemize}
		\item Código derivado: Modelagem do sistema;
	\end{itemize}
\end{itemize}
17- Como você elabora o modelo de dados do sistema (ferramentas, MER, ML)? Se nunca elaborou, considera-o importante?\newline
18- Você valida tecnicamente este modelo com área responsável ou com algum padrão disponível (nomenclatura e formas normais)? Se nunca validou, considera-o importante?\newline
\newline
\textbf{Codificação}\newline\newline
Objetivo(s):
\begin{itemize}
	\item Obter informações sobre a depuração de erros do sistema.
	\begin{itemize}
		\item Código derivado: Depuração;
	\end{itemize}
	\item Obter informações sobre estilos de codificação.
	\begin{itemize}
		\item Código derivado: Estilo e \textit{Design} de codificação;
	\end{itemize}
	\item Obter informações sobre o ambiente de codificação.
	\begin{itemize}
		\item Código derivado: Ambiente de codificação;
	\end{itemize}
\end{itemize}
19- Você costuma utilizar código PL/SQL, HTML e Javascript em mais da metade da aplicação?\newline
20- Você procurar escrever os seus códigos PL/SQL, SQL, HTML e Javascript de forma legível?\newline
21- Você segue algum padrão de codificação ou boas práticas (constantes, nomes significativos) para PL/SQL, SQL, HTML e Javascript?\newline
22- Você costuma usar algum editor de código ou o próprio APEX, para escrever seus códigos?\newline
23- Você costuma depurar o código?\newline
24- Você costuma classificar os erros encontrados (falha de execução ou comportamento inesperado), de forma que possa ajudá-lo na abordagem de resolução?\newline
25- Você costuma comentar o código que faz?\newline
\newline\newline
\textbf{Teste}\newline\newline
Objetivo(s):
\begin{itemize}
	\item Obter informações sobre a realização dos testes, bem como a forma de realizá-los.
	\begin{itemize}
		\item Código derivado: Teste;
	\end{itemize}
\end{itemize}
26- Como você testa as suas aplicações?\newline
27- Você costuma testar todas as páginas de aplicação que desenvolve?\newline
28- Você costuma armazenar os resultados dos testes que realiza?\newline
29- Você costuma a elaborar casos de testes para sua aplicação?\newline
30- Seria bom testes automatizados de interface para sua aplicação (selenium)?\newline
\newline
\textbf{Implantação}\newline\newline
Objetivo(s):
\begin{itemize}
	\item Obter informações sobre a migração, para a produção, do sistema.
	\begin{itemize}
		\item Código derivado: Implantação;
	\end{itemize}
\end{itemize}
31- Você costuma migrar as definições dos objetos de banco (tabelas, sequences, views e etc) junto com a aplicação?\newline
32- Você costuma verificar se os apelidos da aplicação de desenvolvimento e produção estão iguais?\newline
33- Realiza uma navegação por todas as páginas da aplicação?\newline
34- Verifica se todas as tabelas foram criadas?\newline
\newline
\chapter{Respostas das Entrevistas}

%Inicio do Desenvolvedor 1
\begin{comment}
\begin{itemize}
\item \textbf{Desenvolvedor 1}
\end{itemize}

\textbf{Descrição do Desenvolvedor}

\begin{enumerate}
\item Qual é o curso que você faz?\newline
- Resposta 1
\item Quantos semestres são o curso?\newline
- Resposta 2
\item Está em qual semestre do curso?\newline
- Resposta 3
\item Qual é o nome da instituição de ensino?\newline
- Resposta 4
\item Quanto tempo de estágio?\newline
- Resposta 5
\end{enumerate}

\textbf{Papel do desenvolvedor na organização}

\begin{enumerate}
\setcounter{enumi}{5}
\item Como você enxerga o seu papel na instituição? (A posição dentro da instituição, as
responsabilidades, e etc. . . )\newline
- Resposta 1
\item Você considera a sua contribuição relevante para a instituição?\newline
- Resposta 2
\end{enumerate}

\textbf{Efetividade do curso}

\begin{enumerate}
\setcounter{enumi}{7}
\item Você acha que o curso o preparou bem para as tarefas que realizou?\newline
- Resposta 1

\item Qual a sua opinião com relação a efetividade do curso?\newline
- Resposta 2
\item Como você avalia a qualidade das aplicações em que deu manutenção?\newline
- Resposta 3
\end{enumerate}

\textbf{Boas práticas e padrões}

\begin{enumerate}
\setcounter{enumi}{10}
\item Você adota algum padrão ou alguma boa prática no desenvolvimento/manutenção das
aplicações?\newline
- Resposta 1
\item Teria alguma sugestão de boa prática/padrão que realiza que poderia melhorar de
alguma forma a qualidade das aplicações desenvolvidas?\newline
- Resposta 2
\end{enumerate}

\textbf{Requisitos}

\begin{enumerate}
\setcounter{enumi}{12}
\item Como você faz para levantar os requisitos do sistema que foi solicitado por alguma
unidade?\newline
- Resposta 1
\item Como você aborda o cliente para levantar os requisitos do sistema?\newline
- Resposta 2
\item Como você armazena ou documenta o que foi passado pelo cliente (requisitos)? Existe
algum meio de armazenar eles diretamente no sistema?\newline
- Resposta 3
\item Você costuma fazer reunião com o cliente em outras etapas do desenvolvimento, para
levantar outros requisitos e/ou esclarecer os que já existem?\newline
- Resposta 4
\end{enumerate}

\textbf{\textit{Design}}

\begin{enumerate}
\setcounter{enumi}{16}
\item Como você elabora o modelo de dados do sistema (ferramentas, MER, ML)? Se nunca
elaborou, considera-o importante?\newline
- Resposta 1
\item Você valida tecnicamente este modelo com área responsável ou com algum padrão
disponível (nomenclatura e formas normais)? Se nunca validou, considera-o importante?\newline
- Resposta 2
\end{enumerate}

\textbf{Codificação}

\begin{enumerate}
\setcounter{enumi}{18}
\item Você costuma utilizar código PL/SQL, HTML e Javascript em mais da metade da
aplicação?\newline
- Resposta 1
\item Você procurar escrever os seus códigos PL/SQL, SQL, HTML e Javascript de forma
legível?\newline
- Resposta 2
\item Você segue algum padrão de codificação ou boas práticas (constantes, nomes signifi-
cativos) para PL/SQL, SQL, HTML e Javascript?\newline
- Resposta 3
\item Você costuma usar algum editor de código ou o próprio APEX, para escrever seus
códigos?\newline
- Resposta 4
\item Você costuma depurar o código?\newline
- Resposta 5
\item Você costuma classificar os erros encontrados (falha de execução ou comportamento
inesperado), de forma que possa ajudá-lo na abordagem de resolução?\newline
- Resposta 6
\item Você costuma comentar o código que faz?\newline
- Resposta 7
\end{enumerate}

\textbf{Teste}

\begin{enumerate}
\setcounter{enumi}{25}
\item Como você testa as suas aplicações?\newline
- Resposta 1
\item Você costuma testar todas as páginas de aplicação que desenvolve?\newline
- Resposta 2
\item Você costuma armazenar os resultados dos testes que realiza?\newline
- Resposta 3
\item Você costuma a elaborar casos de testes para sua aplicação?\newline
- Resposta 4
\item Seria bom testes automatizados de interface para sua aplicação (selenium)?\newline
- Resposta 5
\end{enumerate}

\textbf{Implantação}

\begin{enumerate}
\setcounter{enumi}{30}
\item Você costuma migrar as definições dos objetos de banco (tabelas, sequences, views e
etc) junto com a aplicação?\newline
- Resposta 1
\item Você costuma verificar se os apelidos da aplicação de desenvolvimento e produção
estão iguais?\newline
- Resposta 2
\item Realiza uma navegação por todas as páginas da aplicação?\newline
- Resposta 3
\item Verifica se todas as tabelas foram criadas?\newline
- Resposta 4
\end{enumerate}
\end{comment}
%Fim das respostas do Desenvolvedor


%Inicio respostas do Desenvolvedor 1
\begin{itemize}
	\item \textbf{Desenvolvedor 1}
\end{itemize}

\textbf{Descrição do Desenvolvedor}

\begin{enumerate}
	\item Qual é o curso que você faz?\newline
	- Análise e Desenvolvimento de Sistemas.
	\item Quantos semestres são o curso?\newline
	- 5 semestres.
	\item Está em qual semestre do curso?\newline
	- Terceiro semestre.
	\item Qual é o nome da instituição de ensino?\newline
	- Faculdade Senac.
	\item Quanto tempo de estágio?\newline
	- 1 ano.
\end{enumerate}

\textbf{Papel do desenvolvedor na organização}

\begin{enumerate}
	\setcounter{enumi}{5}
	\item Como você enxerga o seu papel na instituição? (A posição dentro da instituição, as
	responsabilidades, e etc. . . )\newline
	-  O papel está claro. Foi contratado para dar manutenção em uma aplicação da Secretária de Comunicação (SECOM), porém esta alocado fisicamente no SEADE, e também realiza alguns trabalhos para o mesmo, apesar de sua função principal ser atender a Secretária de Comunicação.
	\item Você considera a sua contribuição relevante para a instituição?\newline
	- Sente que sua contribuição é relevante.
\end{enumerate}

\textbf{Efetividade do Treinamento}

\begin{enumerate}
	\setcounter{enumi}{7}
	\item Você acha que o curso o preparou bem para as tarefas que realizou?\newline
	- Não. O curso somente introduz o APEX, muitas das coisas são vistas somente na prática.
	\item Qual a sua opinião com relação a efetividade do curso?\newline
	- Considera-o importante para apresentar a ferramenta, porém muitas coisas não são abordadas.
	\item Como você avalia a qualidade das aplicações em que deu manutenção?\newline
	- A aplicação em que deu manutenção algumas partes estavam muito bem detalhadas e com bastante comentários, porém outras partes estavam com coisas que não estavam mais sendo utilizadas, que deveriam ser removidas. Segundo o entrevistado, parece ser consequência da aplicação ter passado por vários desenvolvedores diferentes.
\end{enumerate}

\textbf{Boas práticas e padrões}

\begin{enumerate}
	\setcounter{enumi}{10}
	\item Você adota algum padrão ou alguma boa prática no desenvolvimento/manutenção das
	aplicações?\newline
	- Mantem códigos que não tem costume de desenvolver com frequência em um arquivo de texto separado, para que possa ser reutilizado em futuros códigos seguindo a mesma lógica. Também costuma comentar o código, apesar de ter adotado essa prática somente depois de um certo tempo de desenvolvimento.
	\item Teria alguma sugestão de boa prática/padrão que realiza que poderia melhorar de
	alguma forma a qualidade das aplicações desenvolvidas?\newline
	- Usar o campo de comentário nas páginas das aplicações APEX, de forma a facilitar o entendimento de outros desenvolvedores.
\end{enumerate}

\textbf{Requisitos}

\begin{enumerate}
	\setcounter{enumi}{12}
	\item Como você faz para levantar os requisitos do sistema que foi solicitado por alguma
	unidade?\newline
	- A maioria das reuniões a unidade é quem marca com o desenvolvedor para levantar os requisitos junto ao usuário.
	\item Como você aborda o cliente para levantar os requisitos do sistema?\newline
	- Via reuniões.
	\item Como você armazena ou documenta o que foi passado pelo cliente (requisitos)? Existe
	algum meio de armazenar eles diretamente no sistema?\newline
	- Armazena inicialmente em um papel e depois passa para o sistema.
	\item Você costuma fazer reunião com o cliente em outras etapas do desenvolvimento, para
	levantar outros requisitos e/ou esclarecer os que já existem?\newline
	- Sim, realiza reuniões durante o desenvolvimento.
\end{enumerate}

\textbf{\textit{Design}}

\begin{enumerate}
	\setcounter{enumi}{16}
	\item Como você elabora o modelo de dados do sistema (ferramentas, MER, ML)? Se nunca
	elaborou, considera-o importante?\newline
	- Não atualiza o modelo de dados. Considera-o importante.
	\item Você valida tecnicamente este modelo com área responsável ou com algum padrão
	disponível (nomenclatura e formas normais)? Se nunca validou, considera-o importante?\newline
	- Não valida o modelo de dados. Considera-o importante.
\end{enumerate}

\textbf{Codificação}

\begin{enumerate}
	\setcounter{enumi}{18}
	\item Você costuma utilizar código PL/SQL, HTML e Javascript em mais da metade da
	aplicação?\newline
	- Menos da metade da aplicação que dava manutenção tinha código, só em ocasiões especiais.
	\item Você procurar escrever os seus códigos PL/SQL, SQL, HTML e Javascript de forma
	legível?\newline
	- Costuma, as vezes, escrever de forma legível.
	\item Você segue algum padrão de codificação ou boas práticas (constantes, nomes signifi-
	cativos) para PL/SQL, SQL, HTML e Javascript?\newline
	- Sim, usa nomes significativos.
	\item Você costuma usar algum editor de código ou o próprio APEX, para escrever seus
	códigos?\newline
	- Usa o editor notepad++, porque os arquivos podem ficar salvos também em outro lugar.
	\item Você costuma depurar o código?\newline
	- Costuma depurar a aplicação, usando a funcionalidade de depurar.
	\item Você costuma classificar os erros encontrados (falha de execução ou comportamento
	inesperado), de forma que possa ajudá-lo na abordagem de resolução?\newline
	- A classificação de erros é somente de forma mental.
	\item Você costuma comentar o código que faz?\newline
	- Comenta as vezes.
\end{enumerate}

\textbf{Teste}

\begin{enumerate}
	\setcounter{enumi}{25}
	\item Como você testa as suas aplicações?\newline
	- Testa as aplicações usando testes funcionais.
	\item Você costuma testar todas as páginas de aplicação que desenvolve?\newline
	- Sim, testa todas as páginas que desenvolve.
	\item Você costuma armazenar os resultados dos testes que realiza?\newline
	- Não armazena os resultados dos testes.
	\item Você costuma a elaborar casos de testes para sua aplicação?\newline
	- Não elabora casos de teste, o faz de forma mais intuitiva.
	\item Seria bom testes automatizados de interface para sua aplicação (selenium)?\newline
	- Acha importante teste automatizado de interface, principalmente nos casos com muitos dados a serem setados que tem chances de erro.
\end{enumerate}

\textbf{Implantação}

\begin{enumerate}
	\setcounter{enumi}{30}
	\item Você costuma migrar as definições dos objetos de banco (tabelas, sequences, views e
	etc) junto com a aplicação?\newline
	- Não. Costuma anotar as alterações feitas e realiza essas alterações no espaço de produção.
	\item Você costuma verificar se os apelidos da aplicação de desenvolvimento e produção
	estão iguais?\newline
	- No começo não. Agora costuma pensar mais na troca de apelidos, após os avisos dos alertas.
	\item Realiza uma navegação por todas as páginas da aplicação?\newline
	- Navegação feita geralmente mais nas páginas que altera.
	\item Verifica se todas as tabelas foram criadas?\newline
	- Verifica a criação das tabelas.
\end{enumerate}

\textbf{Observações}

\begin{itemize}
	\item Se tivesse uma ferramente que mostrasse as diferenças entre os modelos de dados dos espaços de desenvolvimento e de produção seria interessante.
	\item Requisitos muito óbvios o desenvolvedor não anotava na reunião, e depois acabava esquecendo eles.
	\item Existe a possibilidade de gerar o modelo de dados a partir das tabelas construídas.
\end{itemize}
%Fim respostas Desenvolvedor 1

%Inicio respostas Desenvolvedor 2

\begin{itemize}
	\item \textbf{Desenvolvedor 2}
\end{itemize}

\textbf{Descrição do Desenvolvedor}

\begin{enumerate}
	\item Qual é o curso que você faz?\newline
	- Sistemas de informação.
	\item Quantos semestres são o curso?\newline
	- 8 semestres.
	\item Está em qual semestre do curso?\newline
	- Nono semestre, porém em termos de conteúdo esta no sexto.
	\item Qual é o nome da instituição de ensino?\newline
	- Universidade Católica de Brasília.
	\item Quanto tempo de estágio?\newline
	- 1 ano.
\end{enumerate}

\textbf{Papel do desenvolvedor na organização}

\begin{enumerate}
	\setcounter{enumi}{5}
	\item Como você enxerga o seu papel na instituição? (A posição dentro da instituição, as
	responsabilidades, e etc. . . )\newline
	- Entende o seu papel no sistema, porém ao terminar suas tarefas no sistema fica perdido em o que fazer em seguida.
	\item Você considera a sua contribuição relevante para a instituição?\newline
	- Sim. Da idéias e sugestões aos servidores, e algumas são escritas como tutoriais na \textit{Wiki}.
\end{enumerate}

\textbf{Efetividade do Treinamento}

\begin{enumerate}
	\setcounter{enumi}{7}
	\item Você acha que o curso o preparou bem para as tarefas que realizou?\newline
	- Não. O curso somente da uma introdução, porém necessita de mais aprofundamento.
	\item Qual a sua opinião com relação a efetividade do curso?\newline
	- Poderia melhorar o curso um pouco mais.
	\item Como você avalia a qualidade das aplicações em que deu manutenção?\newline
	- Entre regular e bom, pois algumas coisas são feitas sem se preocupar com manutenções futuras.
\end{enumerate}

\textbf{Boas práticas e padrões}

\begin{enumerate}
	\setcounter{enumi}{10}
	\item Você adota algum padrão ou alguma boa prática no desenvolvimento/manutenção das
	aplicações?\newline
	- Documentar o código.
	\item Teria alguma sugestão de boa prática/padrão que realiza que poderia melhorar de
	alguma forma a qualidade das aplicações desenvolvidas?\newline
	- Construir uma API APEX, com informações que podem ser utilizadas no órgão, como: procedimento para obter o código de uma unidade, o e-mail do chefe de serviço, como implementar funcionalidades fora do comum, e etc. A idéia é divulgar o conhecimento. Atualmente esta em processo de definição.
\end{enumerate}

\textbf{Requisitos}

\begin{enumerate}
	\setcounter{enumi}{12}
	\item Como você faz para levantar os requisitos do sistema que foi solicitado por alguma
	unidade?\newline
	- Reunião junto com o consultor técnico no início. Após um tempo, o desenvolvedor começou a marcar as reuniões diretamente com os interessados.
	\item Como você aborda o cliente para levantar os requisitos do sistema?\newline
	- Através de reuniões.
	\item Como você armazena ou documenta o que foi passado pelo cliente (requisitos)? Existe
	algum meio de armazenar eles diretamente no sistema?\newline
	- Anota em papel quando vai nas reuniões. Após implementar os requisitos, passa eles para o DGA.
	\item Você costuma fazer reunião com o cliente em outras etapas do desenvolvimento, para
	levantar outros requisitos e/ou esclarecer os que já existem?\newline
	- Não, pois realiza primeiro a implementação no sistema e logo após é que confirma se os requisitos estão corretos, junto ao cliente.
\end{enumerate}

\textbf{\textit{Design}}

\begin{enumerate}
	\setcounter{enumi}{16}
	\item Como você elabora o modelo de dados do sistema (ferramentas, MER, ML)? Se nunca
	elaborou, considera-o importante?\newline
	- Não atualiza o modelo de dados, pois ele esta muito desatualizado e necessitaria de um esforço grande pra deixar ele refletindo o estado atual do sistema.
	\item Você valida tecnicamente este modelo com área responsável ou com algum padrão
	disponível (nomenclatura e formas normais)? Se nunca validou, considera-o importante?\newline
	- Nunca validou o modelo de dados. Considera-o importante.
\end{enumerate}

\textbf{Codificação}

\begin{enumerate}
	\setcounter{enumi}{18}
	\item Você costuma utilizar código PL/SQL, HTML e Javascript em mais da metade da
	aplicação?\newline
	- Mais da metade da aplicação que trabalhou possui código.
	\item Você procurar escrever os seus códigos PL/SQL, SQL, HTML e Javascript de forma
	legível?\newline
	- Sim, procura escrever os códigos de forma legível.
	\item Você segue algum padrão de codificação ou boas práticas (constantes, nomes signifi-
	cativos) para PL/SQL, SQL, HTML e Javascript?\newline
	- Sim, identação e padrões de nomenclatura.
	\item Você costuma usar algum editor de código ou o próprio APEX, para escrever seus
	códigos?\newline
	- Usa o editor notepad++.
	\item Você costuma depurar o código?\newline
	- Costuma depurar a aplicação, usando a funcionalidade de depurar.
	\item Você costuma classificar os erros encontrados (falha de execução ou comportamento
	inesperado), de forma que possa ajudá-lo na abordagem de resolução?\newline
	- Não classifica os erros encontrados.
	\item Você costuma comentar o código que faz?\newline
	- Sim, costuma comentar bastante o código.
\end{enumerate}

\textbf{Teste}

\begin{enumerate}
	\setcounter{enumi}{25}
	\item Como você testa as suas aplicações?\newline
	- Testa a aplicação no ambiente de teste. Após os testes passarem, leva para o ambiente de produção. No ambiente de produção, faz uma homologação com o usuário antes de liberar de forma definitiva para a produção. Desta forma, a aplicação atual continua funcionando, e a nova vai sendo validada pelo usuário. Qualquer ajuste que precisa ser feito não interfere na aplicação atual. Após a validação do usuário, disponibiliza definitivamente para o uso.
	\item Você costuma testar todas as páginas de aplicação que desenvolve?\newline
	- Sim, testa todas as páginas que desenvolve.
	\item Você costuma armazenar os resultados dos testes que realiza?\newline
	- Não armazena os resultados dos testes.
	\item Você costuma a elaborar casos de testes para sua aplicação?\newline
	- Tenta prever de forma mais intuitiva os casos de teste.
	\item Seria bom testes automatizados de interface para sua aplicação (selenium)?\newline
	- Acha que seria benéfico teste automatizado de interface.
\end{enumerate}

\textbf{Implantação}

\begin{enumerate}
	\setcounter{enumi}{30}
	\item Você costuma migrar as definições dos objetos de banco (tabelas, sequences, views e
	etc) junto com a aplicação?\newline
	- Nunca precisou fazer migração de objetos de banco.
	\item Você costuma verificar se os apelidos da aplicação de desenvolvimento e produção
	estão iguais?\newline
	- Sim, costuma verificar os apelidos. O alerta ajudou a se preocupar com isso.
	\item Realiza uma navegação por todas as páginas da aplicação?\newline
	- Navega somente nas páginas que fazem parte do fluxo relacionado com a funcionalidade.
	\item Verifica se todas as tabelas foram criadas?\newline
	- Verifica se as tabelas foram criadas.
\end{enumerate}

\textbf{Observação}

\begin{itemize}
	\item Por conta de muita demanda, alguns pontos podem acabar sendo esquecidos de testar.
	\item Faz homologação das aplicações junto ao cliente.
\end{itemize}

%Fim respostas Desenvolvedor 2

%Inicio respostas do Desenvolvedor 3
\begin{itemize}
	\item \textbf{Desenvolvedor 3}
\end{itemize}

\textbf{Descrição do Desenvolvedor}

\begin{enumerate}
	\item Qual é o curso que você faz?\newline
	- Sistemas de Informação.
	\item Quantos semestres são o curso?\newline
	- 8 semestres.
	\item Está em qual semestre do curso?\newline
	- Sétimo semestre.
	\item Qual é o nome da instituição de ensino?\newline
	- Universidade Católica de Brasília.
	\item Quanto tempo de estágio?\newline
	- 7 meses.
\end{enumerate}

\textbf{Papel do desenvolvedor na organização}

\begin{enumerate}
	\setcounter{enumi}{5}
	\item Como você enxerga o seu papel na instituição? (A posição dentro da instituição, as
	responsabilidades, e etc. . . )\newline
	- O papel esta claro na visão do desenvolvedor, suas responsabilidades estão claras desde quando o mesmo foi contratado.
	\item Você considera a sua contribuição relevante para a instituição?\newline
	- Sim, pois entende que esta contribuindo para uma aplicação da qual a instituição depende.
\end{enumerate}

\textbf{Efetividade do Treinamento}

\begin{enumerate}
	\setcounter{enumi}{7}
	\item Você acha que o curso o preparou bem para as tarefas que realizou?\newline
	- Não, o curso aborda de forma muito superficial o desenvolvimento APEX, ainda mais se o desenvolvedor nunca trabalhou ou ouviu falar do APEX, principalmente com a questão do uso de código na aplicação.
	\item Qual a sua opinião com relação a efetividade do curso?\newline
	- Efetividade fraca.
	\item Como você avalia a qualidade das aplicações em que deu manutenção?\newline
	- Qualidade baixa, aplicação desenvolvida de uma forma não sustentável, dando a impressão de que ela nem chegaria a ser concluída da forma que estava sendo feita, pois usava muito código e pouco componente padrão do APEX.
\end{enumerate}

\textbf{Boas práticas e padrões}

\begin{enumerate}
	\setcounter{enumi}{10}
	\item Você adota algum padrão ou alguma boa prática no desenvolvimento/manutenção das
	aplicações?\newline
	- Sim, usa comentários.
	\item Teria alguma sugestão de boa prática/padrão que realiza que poderia melhorar de
	alguma forma a qualidade das aplicações desenvolvidas?\newline
	- Comentar as aplicações e procurar usar os componentes padrões do APEX ao máximo, evitando o uso de código sempre que possível de forma a evitar problemas a futuros desenvolvedores inexperientes.
\end{enumerate}

\textbf{Requisitos}

\begin{enumerate}
	\setcounter{enumi}{12}
	\item Como você faz para levantar os requisitos do sistema que foi solicitado por alguma
	unidade?\newline
	- Procura reunir todos os interessados, de forma a obter os requisitos. Dúvidas pontuais que surgem, são tiradas com os interessados através de trocas de e-mail.
	\item Como você aborda o cliente para levantar os requisitos do sistema?\newline
	- Através de reuniões e trocas de e-mail.
	\item Como você armazena ou documenta o que foi passado pelo cliente (requisitos)? Existe
	algum meio de armazenar eles diretamente no sistema?\newline
	- Armazena os requisitos levantados no próprio e-mail do desenvolvedor.
	\item Você costuma fazer reunião com o cliente em outras etapas do desenvolvimento, para
	levantar outros requisitos e/ou esclarecer os que já existem?\newline
	- Sim, costuma reunir novamente com o cliente.
\end{enumerate}

\textbf{\textit{Design}}

\begin{enumerate}
	\setcounter{enumi}{16}
	\item Como você elabora o modelo de dados do sistema (ferramentas, MER, ML)? Se nunca
	elaborou, considera-o importante?\newline
	- Não chegou a elaborar o modelo de dados, mas o considera importante.
	\item Você valida tecnicamente este modelo com área responsável ou com algum padrão
	disponível (nomenclatura e formas normais)? Se nunca validou, considera-o importante?\newline
	- Sim, procura validar as alterações feitas no modelo de dados junto a área técnica.
\end{enumerate}

\textbf{Codificação}

\begin{enumerate}
	\setcounter{enumi}{18}
	\item Você costuma utilizar código PL/SQL, HTML e Javascript em mais da metade da
	aplicação?\newline
	- Na maior parte das páginas que desenvolveu, usou algum tipo de código.
	\item Você procurar escrever os seus códigos PL/SQL, SQL, HTML e Javascript de forma
	legível?\newline
	- Sim, sempre pensando no próximo que irá mexer na aplicação.
	\item Você segue algum padrão de codificação ou boas práticas (constantes, nomes signifi-
	cativos) para PL/SQL, SQL, HTML e Javascript?\newline
	- Sim, costuma usar nomes significativos para variáveis.
	\item Você costuma usar algum editor de código ou o próprio APEX, para escrever seus
	códigos?\newline
	- Para códigos grandes costuma usar um editor de código. Para códigos pequenos, costuma escrever/editar no próprio APEX.
	\item Você costuma depurar o código?\newline
	- Não, pois não conhecia muito bem esta funcionalidade.
	\item Você costuma classificar os erros encontrados (falha de execução ou comportamento
	inesperado), de forma que possa ajudá-lo na abordagem de resolução?\newline
	- Não classifica os erros encontrados.
	\item Você costuma comentar o código que faz?\newline
	- Sim, comenta o código.
\end{enumerate}

\textbf{Teste}

\begin{enumerate}
	\setcounter{enumi}{25}
	\item Como você testa as suas aplicações?\newline
	- Solicita ao cliente para testar a aplicação, de forma a coletar \textit{feedback} da experiência de uso (erros e sugestão de melhorias).
	\item Você costuma testar todas as páginas de aplicação que desenvolve?\newline
	- Sim, testa todas as páginas desenvolvidas.
	\item Você costuma armazenar os resultados dos testes que realiza?\newline
	- Não armazena os resultados dos testes.
	\item Você costuma a elaborar casos de testes para sua aplicação?\newline
	- Elabora de forma mental.
	\item Seria bom testes automatizados de interface para sua aplicação (selenium)?\newline
	- Seria interessante dependendo do tamanho da aplicação. O esforço para codificar os testes não seria interessante pra uma aplicação muito pequena. Já para uma aplicação média/grande, seria interessante.
\end{enumerate}

\textbf{Implantação}

\begin{enumerate}
	\setcounter{enumi}{30}
	\item Você costuma migrar as definições dos objetos de banco (tabelas, sequences, views e
	etc) junto com a aplicação?\newline
	- Algumas vezes migra as definições dos objetos.
	\item Você costuma verificar se os apelidos da aplicação de desenvolvimento e produção
	estão iguais?\newline
	- Sim, verifica os apelidos.
	\item Realiza uma navegação por todas as páginas da aplicação?\newline
	- Sim, navega por todas as páginas.
	\item Verifica se todas as tabelas foram criadas?\newline
	- Somente as tabelas que foram alteradas ou criadas, de forma que as que existem acabam dependendo da execução da aplicação para saber se continuam corretas ou não (se a aplicação não quebrou).
\end{enumerate}

\textbf{Observações}
\begin{itemize}
	\item Costuma, de forma proposital, deixar um erro óbvio para verificar se o cliente de fato esta testando a aplicação.
\end{itemize}
%Fim das respostas do Desenvolvedor 3

%Inicio do Desenvolvedor 4

\begin{itemize}
	\item \textbf{Desenvolvedor 4}
\end{itemize}

\textbf{Descrição do Desenvolvedor}

\begin{enumerate}
	\item Qual é o curso que você faz?\newline
	- Análise e Desenvolvimento de Sistemas.
	\item Quantos semestres são o curso?\newline
	- 5 semestres.
	\item Está em qual semestre do curso?\newline
	- Quinto semestre.
	\item Qual é o nome da instituição de ensino?\newline
	- Faculdade Projeção.
	\item Quanto tempo de estágio?\newline
	- 7 meses.
\end{enumerate}

\textbf{Papel do desenvolvedor na organização}

\begin{enumerate}
	\setcounter{enumi}{5}
	\item Como você enxerga o seu papel na instituição? (A posição dentro da instituição, as
	responsabilidades, e etc. . . )\newline
	- Enxerga seu papel de forma clara.
	\item Você considera a sua contribuição relevante para a instituição?\newline
	- Sim, considera sua contribuição relevante.
\end{enumerate}

\textbf{Efetividade do Treinamento}

\begin{enumerate}
	\setcounter{enumi}{7}
	\item Você acha que o curso o preparou bem para as tarefas que realizou?\newline
	- Não, o curso é bem básico.
	\item Qual a sua opinião com relação a efetividade do curso?\newline
	- Poderia melhorar, a demanda das tarefas vai além do que o curso apresenta.
	\item Como você avalia a qualidade das aplicações em que deu manutenção?\newline
	- A aplicação não foi bem construída, pois passou pelas mãos de várias pessoas, o que acabou deixando gambiarras/improvisos no sistema.
\end{enumerate}

\textbf{Boas práticas e padrões}

\begin{enumerate}
	\setcounter{enumi}{10}
	\item Você adota algum padrão ou alguma boa prática no desenvolvimento/manutenção das
	aplicações?\newline
	- Sim, procura construir consultas SQL com filtros, para melhor performance.
	\item Teria alguma sugestão de boa prática/padrão que realiza que poderia melhorar de
	alguma forma a qualidade das aplicações desenvolvidas?\newline
	- Não.
\end{enumerate}

\textbf{Requisitos}

\begin{enumerate}
	\setcounter{enumi}{12}
	\item Como você faz para levantar os requisitos do sistema que foi solicitado por alguma
	unidade?\newline
	- O responsável pelo desenvolvedor marca uma reunião com o cliente e desenvolvedor vai junto para a reunião, onde vai anotando os pontos levantados pelo cliente.
	\item Como você aborda o cliente para levantar os requisitos do sistema?\newline
	- Através de reuniões.
	\item Como você armazena ou documenta o que foi passado pelo cliente (requisitos)? Existe
	algum meio de armazenar eles diretamente no sistema?\newline
	- Primeiramente anota em um papel durante a reunião, e então passa para o documento de Descrição Geral de Aplicação, e armazena o documento no repositório SVN.
	\item Você costuma fazer reunião com o cliente em outras etapas do desenvolvimento, para
	levantar outros requisitos e/ou esclarecer os que já existem?\newline
	- Sim, costuma reunir várias vezes com o cliente.
\end{enumerate}

\textbf{\textit{Design}}

\begin{enumerate}
	\setcounter{enumi}{16}
	\item Como você elabora o modelo de dados do sistema (ferramentas, MER, ML)? Se nunca
	elaborou, considera-o importante?\newline
	- Não chegou a elaborar o modelo de dados, mas o considera importante.
	\item Você valida tecnicamente este modelo com área responsável ou com algum padrão
	disponível (nomenclatura e formas normais)? Se nunca validou, considera-o importante?\newline
	- Sim, algumas vezes valida o modelo de dados.
\end{enumerate}

\textbf{Codificação}

\begin{enumerate}
	\setcounter{enumi}{18}
	\item Você costuma utilizar código PL/SQL, HTML e Javascript em mais da metade da
	aplicação?\newline
	- Mais da metade da aplicação que trabalhou possui código.
	\item Você procurar escrever os seus códigos PL/SQL, SQL, HTML e Javascript de forma
	legível?\newline
	- Sim, procura usar comentários e identação no código.
	\item Você segue algum padrão de codificação ou boas práticas (constantes, nomes signifi-
	cativos) para PL/SQL, SQL, HTML e Javascript?\newline
	- Sim, procura usar comentários, identação, nomes significativos, otimizar as consultas com filtro, e evita usar ao máximo usar funções para não pesar tanto a aplicação.
	\item Você costuma usar algum editor de código ou o próprio APEX, para escrever seus
	códigos?\newline
	- Costuma usar um editor de código.
	\item Você costuma depurar o código?\newline
	- Sim, costuma testar no console de comandos SQL para verificar se o código se comporta como o esperado.
	\item Você costuma classificar os erros encontrados (falha de execução ou comportamento
	inesperado), de forma que possa ajudá-lo na abordagem de resolução?\newline
	- Não classifica os erros.
	\item Você costuma comentar o código que faz?\newline
	- Sim, de forma que ajude também o desenvolvedor futuramente a se lembrar do comportamento do código.
\end{enumerate}

\textbf{Teste}

\begin{enumerate}
	\setcounter{enumi}{25}
	\item Como você testa as suas aplicações?\newline
	- Testa os códigos desenvolvidos no console de comandos SQL, e também testa manualmente a funcionalidade da aplicação.
	\item Você costuma testar todas as páginas de aplicação que desenvolve?\newline
	- Sim, testa todas as páginas desenvolvidas.
	\item Você costuma armazenar os resultados dos testes que realiza?\newline
	- Não, porém alguns erros dependendo da situação podem ser documentados.
	\item Você costuma a elaborar casos de testes para sua aplicação?\newline
	- Elabora os casos de teste mentalmente.
	\item Seria bom testes automatizados de interface para sua aplicação (selenium)?\newline
	- Seria interessante teste automatizado de interface.
\end{enumerate}

\textbf{Implantação}

\begin{enumerate}
	\setcounter{enumi}{30}
	\item Você costuma migrar as definições dos objetos de banco (tabelas, sequences, views e
	etc) junto com a aplicação?\newline
	- Sim, costuma sempre migrar as definições de objeto.
	\item Você costuma verificar se os apelidos da aplicação de desenvolvimento e produção
	estão iguais?\newline
	- Sim, verifica se os apelidos são iguais.
	\item Realiza uma navegação por todas as páginas da aplicação?\newline
	- Sim, realiza uma navegação por todas as páginas.
	\item Verifica se todas as tabelas foram criadas?\newline
	- Sim, verifica todas as tabelas.
\end{enumerate}

\textbf{Observações}

\begin{itemize}
	\item Existir alguma ferramenta que permitisse testar os códigos sem depender do APEX, de forma que o trabalho não seja interrompido caso o APEX venha a cair.
	\item Sempre testar o código desenvolvido no console de comandos SQL, antes de passar para a aplicação APEX.
	\item Usar uma pessoa que não seja o desenvolvedor para testar a aplicação, de forma a ter um teste menos viciado e mais completo.
	\item Maior problema que teve foi com relação a falta de uma documentação detalhada do sistema, devido ao sistema ser grande e complexo. Documentação foi negligenciada em favor do desenvolvimento do sistema.
\end{itemize}

%Fim das respostas do Desenvolvedor 4

%Inicio do Desenvolvedor 5

\begin{itemize}
	\item \textbf{Desenvolvedor 5}
\end{itemize}

\textbf{Descrição do Desenvolvedor}

\begin{enumerate}
	\item Qual é o curso que você faz?\newline
	- Sistemas de informação.
	\item Quantos semestres são o curso?\newline
	- 8 semestres.
	\item Está em qual semestre do curso?\newline
	- Sexto semestre.
	\item Qual é o nome da instituição de ensino?\newline
	- Faculdade Projeção.
	\item Quanto tempo de estágio?\newline
	- 1 ano.
\end{enumerate}

\textbf{Papel do desenvolvedor na organização}

\begin{enumerate}
	\setcounter{enumi}{5}
	\item Como você enxerga o seu papel na instituição? (A posição dentro da instituição, as
	responsabilidades, e etc. . . )\newline
	-  Bastante responsabilidade, pois desenvolve alguns sistemas sozinho e é responsável desde o levantamento dos requisitos até a implantação, foi responsável pelo desenvolvimento do tema de interface que todos os departamentos utilizam em seus sistemas, fatos esses que corroboram para uma responsabilidade grande dentro da instituição.
	\item Você considera a sua contribuição relevante para a instituição?\newline
	- As matérias básicas vista na faculdade são importantes mas boa parte não se aplica para a instituição com exceção de levantamento de requisitos e lógica de programação que se faz relevante a contribuição para a instituição.
\end{enumerate}

\textbf{Efetividade do Treinamento}

\begin{enumerate}
	\setcounter{enumi}{7}
	\item Você acha que o curso o preparou bem para as tarefas que realizou?\newline
	- Não, o curso é fraco.
	\item Qual a sua opinião com relação a efetividade do curso?\newline
	- A efetividade do curso é fraca.
	\item Como você avalia a qualidade das aplicações em que deu manutenção?\newline
	- As aplicações em que deu manutenção estavam bem desenvolvidas não precisando esforço para entendimento do código, pois foram aplicações desenvolvidas por um servidor que já tem conhecimento sobre desenvolvimento e sabe da importância de um sistema ser bem desenvolvido para diminuir a complexidade de futuras manutenções.
\end{enumerate}

\textbf{Boas práticas e padrões}

\begin{enumerate}
	\setcounter{enumi}{10}
	\item Você adota algum padrão ou alguma boa prática no desenvolvimento/manutenção das
	aplicações?\newline
	- Procura adotar os padrões do apex, tentando não inventar novas formas de implementar.
	\item Teria alguma sugestão de boa prática/padrão que realiza que poderia melhorar de
	alguma forma a qualidade das aplicações desenvolvidas?\newline
	- Treinamento de boas práticas voltadas ao APEX e o sistema de desenvolvimento da instituição. Há muitos recursos do APEX que não são passados.
\end{enumerate}

\textbf{Requisitos}

\begin{enumerate}
	\setcounter{enumi}{12}
	\item Como você faz para levantar os requisitos do sistema que foi solicitado por alguma
	unidade?\newline
	- Como ao desenvolver um sistema normalmente a unidade se utiliza de muitas regras de negócio, então utilizava-se de gravações com o usuário durante as entrevistas, para ter os registros de detalhes importantes que normalmente não se conseguem ver de início e também não esquecer os detalhes. Também faz um esboço do que entendeu e tenta se imergir no ambiente, vivendo o negócio para o entender bem.
	\item Como você aborda o cliente para levantar os requisitos do sistema?\newline
	- Com entrevistas.
	\item Como você armazena ou documenta o que foi passado pelo cliente (requisitos)? Existe
	algum meio de armazenar eles diretamente no sistema?\newline
	- Utiliza a ferramenta SVN para guardar os áudios e documentos.
	\item Você costuma fazer reunião com o cliente em outras etapas do desenvolvimento, para
	levantar outros requisitos e/ou esclarecer os que já existem?\newline
	- Sim, realiza reuniões semanalmente. Porém normalmente já está na mesma sala que o cliente, mantendo sempre contato para sanar dúvidas, alterar requisitos, incluir novos requisitos e assim por diante.
\end{enumerate}

\textbf{\textit{Design}}

\begin{enumerate}
	\setcounter{enumi}{16}
	\item Como você elabora o modelo de dados do sistema (ferramentas, MER, ML)? Se nunca
	elaborou, considera-o importante?\newline
	- Elabora primeiramente desenhando na mão, depois utiliza a ferramenta \textit{Workbench} para a modelagem de dados.
	\item Você valida tecnicamente este modelo com área responsável ou com algum padrão
	disponível (nomenclatura e formas normais)? Se nunca validou, considera-o importante?\newline
	- Não valida o modelo de dados.
\end{enumerate}

\textbf{Codificação}

\begin{enumerate}
	\setcounter{enumi}{18}
	\item Você costuma utilizar código PL/SQL, HTML e Javascript em mais da metade da
	aplicação?\newline
	- Usa em mais da metade da aplicação, mas tem ciência de que para o apex isso é ruim, no entanto como precisa entregar o sistema num dado prazo acaba fazendo dessa maneira e corrigindo posteriormente numa manutenção.
	\item Você procurar escrever os seus códigos PL/SQL, SQL, HTML e Javascript de forma
	legível?\newline
	- Sim, pois sabe da importância de ter legibilidade de código.
	\item Você segue algum padrão de codificação ou boas práticas (constantes, nomes signifi-
	cativos) para PL/SQL, SQL, HTML e Javascript?\newline
	- Não, pois não tem um padrão específico da instituição a se seguir, utiliza padrões e boas práticas que julga ser correta para o desenvolvimento.
	\item Você costuma usar algum editor de código ou o próprio APEX, para escrever seus
	códigos?\newline
	- Usa o \textit{sublime} para funções grandes e o próprio apex para funções que não são grandes.
	\item Você costuma depurar o código?\newline
	- Não usa porque não faz parte do escopo.
	\item Você costuma classificar os erros encontrados (falha de execução ou comportamento
	inesperado), de forma que possa ajudá-lo na abordagem de resolução?\newline
	- Não costuma classificar.
	\item Você costuma comentar o código que faz?\newline
	- Pouco, pois procura deixar o código legível, com exceção de códigos \textit{jquery} e \textit{javascript} que costuma comentar bastante.
\end{enumerate}

\textbf{Teste}

\begin{enumerate}
	\setcounter{enumi}{25}
	\item Como você testa as suas aplicações?\newline
	- Testa manualmente inserido dados de entrada e verificando se as saídas são o que se esperava ou comportamento esperado.
	\item Você costuma testar todas as páginas de aplicação que desenvolve?\newline
	- Sim, todas as páginas são testadas várias vezes.
	\item Você costuma armazenar os resultados dos testes que realiza?\newline
	- Não armazena os resultados dos testes.
	\item Você costuma a elaborar casos de testes para sua aplicação?\newline
	- Não elabora casos de teste.
	\item Seria bom testes automatizados de interface para sua aplicação (selenium)?\newline
	- Seria bom.
\end{enumerate}

\textbf{Implantação}

\begin{enumerate}
	\setcounter{enumi}{30}
	\item Você costuma migrar as definições dos objetos de banco (tabelas, sequences, views e
	etc) junto com a aplicação?\newline
	- Realizou o processo de migração de objetos apenas uma vez.
	\item Você costuma verificar se os apelidos da aplicação de desenvolvimento e produção
	estão iguais?\newline
	- Sim, costuma verificar os apelidos.
	\item Realiza uma navegação por todas as páginas da aplicação?\newline
	- Sim, navega por todas as páginas.
	\item Verifica se todas as tabelas foram criadas?\newline
	- Sim verifica as tabelas.
\end{enumerate}

%Fim das respostas do Desenvolvedor 5

%Inicio do Desenvolvedor 6

\begin{itemize}
	\item \textbf{Desenvolvedor 6}
\end{itemize}

\textbf{Descrição do Desenvolvedor}

\begin{enumerate}
	\item Qual é o curso que você faz?\newline
	- Não se aplica.
	\item Quantos semestres são o curso?\newline
	- Não se aplica.
	\item Está em qual semestre do curso?\newline
	- Não se aplica.
	\item Qual é o nome da instituição de ensino?\newline
	- Não se aplica.
	\item Quanto tempo de estágio?\newline
	- Não se aplica.
\end{enumerate}

\textbf{Papel do desenvolvedor na organização}

\begin{enumerate}
	\setcounter{enumi}{5}
	\item Como você enxerga o seu papel na instituição? (A posição dentro da instituição, as
	responsabilidades, e etc. . . )\newline
	-  Grande responsabilidade, pois é o chefe do setor de um dos 3 núcleos centrais de desenvolvimento descentralizado.
	\item Você considera a sua contribuição relevante para a instituição?\newline
	- Muito importante pois vários setores sempre solicitam o trabalho e ocorre bastante demanda para desenvolvimento de sistemas.
\end{enumerate}

\textbf{Efetividade do Treinamento}

\begin{enumerate}
	\setcounter{enumi}{7}
	\item Você acha que o curso o preparou bem para as tarefas que realizou?\newline
	- Não se aplica.
	\item Qual a sua opinião com relação a efetividade do curso?\newline
	- Não se aplica.
	\item Como você avalia a qualidade das aplicações em que deu manutenção?\newline
	- As aplicações foram bem desenvolvidas recebendo normalmente vários elogios, as manutenções que ocorrem são pontuais alterando algumas partes a pedido do usuário.
\end{enumerate}

\textbf{Boas práticas e padrões}

\begin{enumerate}
	\setcounter{enumi}{10}
	\item Você adota algum padrão ou alguma boa prática no desenvolvimento/manutenção das
	aplicações?\newline
	- Adota padrões próprios e alguns padrões sugeridos pela informática.
	\item Teria alguma sugestão de boa prática/padrão que realiza que poderia melhorar de
	alguma forma a qualidade das aplicações desenvolvidas?\newline
	- Padrões de nomenclatura de tabelas e nomes de colunas.
\end{enumerate}

\textbf{Requisitos}

\begin{enumerate}
	\setcounter{enumi}{12}
	\item Como você faz para levantar os requisitos do sistema que foi solicitado por alguma
	unidade?\newline
	- Faz uma reunião com o cliente e já vai desenvolvendo os requisitos obtidos como um protótipo. O protótipo desenvolvido é testado com o cliente, o que diminui muito as chances de desenvolver o sistema de forma errada.
	\item Como você aborda o cliente para levantar os requisitos do sistema?\newline
	- Sempre marca hora, agendando entrevistas com o cliente.
	\item Como você armazena ou documenta o que foi passado pelo cliente (requisitos)? Existe
	algum meio de armazenar eles diretamente no sistema?\newline
	- Utiliza um sistema chamado documentador para armazenamento das informações.
	\item Você costuma fazer reunião com o cliente em outras etapas do desenvolvimento, para
	levantar outros requisitos e/ou esclarecer os que já existem?\newline
	- Sim, realiza várias outras reuniões para ir mostrando como o sistema está ficando e trabalhando em cima do que já foi feito para chegar ao ponto que o cliente deseja.
\end{enumerate}

\textbf{\textit{Design}}

\begin{enumerate}
	\setcounter{enumi}{16}
	\item Como você elabora o modelo de dados do sistema (ferramentas, MER, ML)? Se nunca
	elaborou, considera-o importante?\newline
	- Elabora, utilizando a ferramenta \textit{DataModeler} para criar o modelo e posteriormente gerar os \textit{scripts} para criação do banco de dados.
	\item Você valida tecnicamente este modelo com área responsável ou com algum padrão
	disponível (nomenclatura e formas normais)? Se nunca validou, considera-o importante?\newline
	- Não precisou validar o modelo de dados.
\end{enumerate}

\textbf{Codificação}

\begin{enumerate}
	\setcounter{enumi}{18}
	\item Você costuma utilizar código PL/SQL, HTML e Javascript em mais da metade da
	aplicação?\newline
	- Para javascript é muito pouco, pois procura seguir o que o apex já proporciona, até por questão de manutenção posterior, já os outros PLSQL e HTML usa em mais da metade.
	\item Você procurar escrever os seus códigos PL/SQL, SQL, HTML e Javascript de forma
	legível?\newline
	- Sim, respeitando o padrão da informática de nomenclatura.
	\item Você segue algum padrão de codificação ou boas práticas (constantes, nomes signifi-
	cativos) para PL/SQL, SQL, HTML e Javascript?\newline
	- Sim, segue o padrão da informática.
	\item Você costuma usar algum editor de código ou o próprio APEX, para escrever seus
	códigos?\newline
	- Usa o \textit{sqldeveloper} para códigos grandes para apontar possíveis erros.
	\item Você costuma depurar o código?\newline
	- Depura mas com pouca frequência.
	\item Você costuma classificar os erros encontrados (falha de execução ou comportamento
	inesperado), de forma que possa ajudá-lo na abordagem de resolução?\newline
	- Não classifica.
	\item Você costuma comentar o código que faz?\newline
	- Não costuma comentar.
\end{enumerate}

\textbf{Teste}

\begin{enumerate}
	\setcounter{enumi}{25}
	\item Como você testa as suas aplicações?\newline
	- Testa manualmente navegando nas páginas.
	\item Você costuma testar todas as páginas de aplicação que desenvolve?\newline
	- Sim, testa todas as páginas.
	\item Você costuma armazenar os resultados dos testes que realiza?\newline
	- Não.
	\item Você costuma a elaborar casos de testes para sua aplicação?\newline
	- Não elabora casos de teste.
	\item Seria bom testes automatizados de interface para sua aplicação (selenium)?\newline
	- Não soube opinar.
\end{enumerate}

\textbf{Implantação}

\begin{enumerate}
	\setcounter{enumi}{30}
	\item Você costuma migrar as definições dos objetos de banco (tabelas, sequences, views e
	etc) junto com a aplicação?\newline
	- Sim, faz a migração dos objetos.
	\item Você costuma verificar se os apelidos da aplicação de desenvolvimento e produção
	estão iguais?\newline
	- Sim, verifica os apelidos.
	\item Realiza uma navegação por todas as páginas da aplicação?\newline
	- Sim, navega por todas as páginas.
	\item Verifica se todas as tabelas foram criadas?\newline
	- Sim verifica a criação das tabelas.
\end{enumerate}

%Fim das respostas do Desenvolvedor 6

%Inicio do Desenvolvedor 7

\begin{itemize}
	\item \textbf{Desenvolvedor 7}
\end{itemize}

\textbf{Descrição do Desenvolvedor}

\begin{enumerate}
	\item Qual é o curso que você faz?\newline
	-Engenharia de Software.
	\item Quantos semestres são o curso?\newline
	- 10 semestres.
	\item Está em qual semestre do curso?\newline
	- Sétimo semestre.
	\item Qual é o nome da instituição de ensino?\newline
	- Universidade de Brasília.
	\item Quanto tempo de estágio?\newline
	- 1 mês.
\end{enumerate}

\textbf{Papel do desenvolvedor na organização}

\begin{enumerate}
	\setcounter{enumi}{5}
	\item Como você enxerga o seu papel na instituição? (A posição dentro da instituição, as
	responsabilidades, e etc. . . )\newline
	- Vejo como um prestador de serviços na área de tecnologia.
	\item Você considera a sua contribuição relevante para a instituição?\newline
	- Sim, considera importante.
\end{enumerate}

\textbf{Efetividade do Treinamento}

\begin{enumerate}
	\setcounter{enumi}{7}
	\item Você acha que o curso o preparou bem para as tarefas que realizou?\newline
	- Sim, porque o curso aborda as principais funcionalidades do apex.
	\item Qual a sua opinião com relação a efetividade do curso?\newline
	- Foi efetivo para criar e dar manutenção nos sistemas feitos em apex.
	\item Como você avalia a qualidade das aplicações em que deu manutenção?\newline
	- Considera de boa qualidade, somente quando continha códigos sql, era complicado dar manutenção.
\end{enumerate}

\textbf{Boas práticas e padrões}

\begin{enumerate}
	\setcounter{enumi}{10}
	\item Você adota algum padrão ou alguma boa prática no desenvolvimento/manutenção das
	aplicações?\newline
	- Não.
	\item Teria alguma sugestão de boa prática/padrão que realiza que poderia melhorar de
	alguma forma a qualidade das aplicações desenvolvidas?\newline
	- Comentários nos códigos sql, plsql e javascript.
\end{enumerate}

\textbf{Requisitos}

\begin{enumerate}
	\setcounter{enumi}{12}
	\item Como você faz para levantar os requisitos do sistema que foi solicitado por alguma
	unidade?\newline
	- Entrevista com o cliente.
	\item Como você aborda o cliente para levantar os requisitos do sistema?\newline
	- Geralmente conversamos sobre o sistema e documentamos as ideias no Descrição Geral da Aplicação (DGA).
	\item Como você armazena ou documenta o que foi passado pelo cliente (requisitos)? Existe
	algum meio de armazenar eles diretamente no sistema?\newline
	- Criamos um documento no Microsoft Word e compartilhamos na rede interna.
	\item Você costuma fazer reunião com o cliente em outras etapas do desenvolvimento, para
	levantar outros requisitos e/ou esclarecer os que já existem?\newline
	- Sim.
\end{enumerate}

\textbf{\textit{Design}}

\begin{enumerate}
	\setcounter{enumi}{16}
	\item Como você elabora o modelo de dados do sistema (ferramentas, MER, ML)? Se nunca
	elaborou, considera-o importante?\newline
	- Fazemos o modelo de dados por meio de softwares disponíveis no órgão, ex: \textit{Data modeler}, \textit{mysql workbench}, \textit{astah} e etc.
	\item Você valida tecnicamente este modelo com área responsável ou com algum padrão
	disponível (nomenclatura e formas normais)? Se nunca validou, considera-o importante?\newline
	- Sim, todos os documentos são validados pelo servidor responsável.
\end{enumerate}

\textbf{Codificação}

\begin{enumerate}
	\setcounter{enumi}{18}
	\item Você costuma utilizar código PL/SQL, HTML e Javascript em mais da metade da
	aplicação?\newline
	- Sim, em mais da metade.
	\item Você procurar escrever os seus códigos PL/SQL, SQL, HTML e Javascript de forma
	legível?\newline
	- Sim, tentando aplicar boas práticas que julga como corretas.
	\item Você segue algum padrão de codificação ou boas práticas (constantes, nomes signifi-
	cativos) para PL/SQL, SQL, HTML e Javascript?\newline
	- Sim, o padrão que julga o correto.
	\item Você costuma usar algum editor de código ou o próprio APEX, para escrever seus
	códigos?\newline
	- O próprio APEX.
	\item Você costuma depurar o código?\newline
	- Raramente.
	\item Você costuma classificar os erros encontrados (falha de execução ou comportamento
	inesperado), de forma que possa ajudá-lo na abordagem de resolução?\newline
	- Não realiza depuração.
	\item Você costuma comentar o código que faz?\newline
	- Não utiliza comentários.
\end{enumerate}

\textbf{Teste}

\begin{enumerate}
	\setcounter{enumi}{25}
	\item Como você testa as suas aplicações?\newline
	- Não chegou ainda a etapa de testes.
	\item Você costuma testar todas as páginas de aplicação que desenvolve?\newline
	- Pretende testar todas.
	\item Você costuma armazenar os resultados dos testes que realiza?\newline
	- Não chegou ainda a etapa de testes.
	\item Você costuma a elaborar casos de testes para sua aplicação?\newline
	- Não chegou ainda a etapa de testes.
	\item Seria bom testes automatizados de interface para sua aplicação (selenium)?\newline
	- Sim, seria de grande ajuda.
\end{enumerate}

\textbf{Implantação}

\begin{enumerate}
	\setcounter{enumi}{30}
	\item Você costuma migrar as definições dos objetos de banco (tabelas, sequences, views e
	etc) junto com a aplicação?\newline
	- Ainda não finalizou nenhuma aplicação, mas pretende.
	\item Você costuma verificar se os apelidos da aplicação de desenvolvimento e produção
	estão iguais?\newline
	- Não realizou migração ainda.
	\item Realiza uma navegação por todas as páginas da aplicação?\newline
	- Sim, realiza navegação por todas as páginas.
	\item Verifica se todas as tabelas foram criadas?\newline
	- Sim, verifica todas as tabelas.
\end{enumerate}

%Fim das respostas do Desenvolvedor 7

%Inicio do Desenvolvedor 8

\begin{itemize}
	\item \textbf{Desenvolvedor 8}
\end{itemize}

\textbf{Descrição do Desenvolvedor}

\begin{enumerate}
	\item Qual é o curso que você faz?\newline
	-Engenharia de Redes de comunicação.
	\item Quantos semestres são o curso?\newline
	- 10 semestres.
	\item Está em qual semestre do curso?\newline
	- Décimo semestre.
	\item Qual é o nome da instituição de ensino?\newline
	- Universidade de Brasília.
	\item Quanto tempo de estágio?\newline
	- 6 meses.
\end{enumerate}

\textbf{Papel do desenvolvedor na organização}

\begin{enumerate}
	\setcounter{enumi}{5}
	\item Como você enxerga o seu papel na instituição? (A posição dentro da instituição, as
	responsabilidades, e etc. . . )\newline
	- Vejo como profissional atuante na parte técnica do tribunal, e o trabalho que eu faço julgo como de suma importância na automatização dos trabalhos críticos da instituição.
	\item Você considera a sua contribuição relevante para a instituição?\newline
	- Sim, pois o trabalho facilita e acelera a produtividade de outros servidores.
\end{enumerate}

\textbf{Efetividade do Treinamento}

\begin{enumerate}
	\setcounter{enumi}{7}
	\item Você acha que o curso o preparou bem para as tarefas que realizou?\newline
	-Não, porque o curso não condiz com os desafios do trabalho.
	\item Qual a sua opinião com relação a efetividade do curso?\newline
	- Considera fraco, a maior parte se aprende com a prática durante o desenvolvimento.
	\item Como você avalia a qualidade das aplicações em que deu manutenção?\newline
	- Acha bem estruturadas, pois foram desenvolvidas por um profissional já conhecedor de boas práticas.
\end{enumerate}

\textbf{Boas práticas e padrões}

\begin{enumerate}
	\setcounter{enumi}{10}
	\item Você adota algum padrão ou alguma boa prática no desenvolvimento/manutenção das
	aplicações?\newline
	- Usa alguns padrões próprios de codificação.
	\item Teria alguma sugestão de boa prática/padrão que realiza que poderia melhorar de
	alguma forma a qualidade das aplicações desenvolvidas?\newline
	- Sim, usar mais as funções pré-estabelecidas do apex.
\end{enumerate}

\textbf{Requisitos}

\begin{enumerate}
	\setcounter{enumi}{12}
	\item Como você faz para levantar os requisitos do sistema que foi solicitado por alguma
	unidade?\newline
	- Reunião com o cliente.
	\item Como você aborda o cliente para levantar os requisitos do sistema?\newline
	- Tenta ver como está a organização dos dados atualmente, tenta discutir melhoras da organização dos dados com o próprio cliente.
	\item Como você armazena ou documenta o que foi passado pelo cliente (requisitos)? Existe
	algum meio de armazenar eles diretamente no sistema?\newline
	- Armazena no próprio computador.
	\item Você costuma fazer reunião com o cliente em outras etapas do desenvolvimento, para
	levantar outros requisitos e/ou esclarecer os que já existem?\newline
	- Sim, com mais reuniões.
\end{enumerate}

\textbf{\textit{Design}}

\begin{enumerate}
	\setcounter{enumi}{16}
	\item Como você elabora o modelo de dados do sistema (ferramentas, MER, ML)? Se nunca
	elaborou, considera-o importante?\newline
	- Através da ferramenta \textit{datamodeler}.
	\item Você valida tecnicamente este modelo com área responsável ou com algum padrão
	disponível (nomenclatura e formas normais)? Se nunca validou, considera-o importante?\newline
	- Sim, com o supervisor.
\end{enumerate}

\textbf{Codificação}

\begin{enumerate}
	\setcounter{enumi}{18}
	\item Você costuma utilizar código PL/SQL, HTML e Javascript em mais da metade da
	aplicação?\newline
	- Sim, em mais da metade.
	\item Você procurar escrever os seus códigos PL/SQL, SQL, HTML e Javascript de forma
	legível?\newline
	- Sim, considera bastante legível.
	\item Você segue algum padrão de codificação ou boas práticas (constantes, nomes signifi-
	cativos) para PL/SQL, SQL, HTML e Javascript?\newline
	- Sim, padrão passado pelo setor.
	\item Você costuma usar algum editor de código ou o próprio APEX, para escrever seus
	códigos?\newline
	- O próprio APEX.
	\item Você costuma depurar o código?\newline
	- Não, não utiliza depuração.
	\item Você costuma classificar os erros encontrados (falha de execução ou comportamento
	inesperado), de forma que possa ajudá-lo na abordagem de resolução?\newline
	- Não classifica.
	\item Você costuma comentar o código que faz?\newline
	- Sim, realiza comentários.
\end{enumerate}

\textbf{Teste}

\begin{enumerate}
	\setcounter{enumi}{25}
	\item Como você testa as suas aplicações?\newline
	- Não costuma testar as aplicações com teste funcional.
	\item Você costuma testar todas as páginas de aplicação que desenvolve?\newline
	- Sim, fazendo navegação.
	\item Você costuma armazenar os resultados dos testes que realiza?\newline
	- Não realiza armazenamento de testes.
	\item Você costuma a elaborar casos de testes para sua aplicação?\newline
	- Sim, de forma mental.
	\item Seria bom testes automatizados de interface para sua aplicação (selenium)?\newline
	- Seria bom, pois melhoraria a qualidade.
\end{enumerate}

\textbf{Implantação}

\begin{enumerate}
	\setcounter{enumi}{30}
	\item Você costuma migrar as definições dos objetos de banco (tabelas, sequences, views e
	etc) junto com a aplicação?\newline
	- Sim, normalmente faz-se a migração de definições.
	\item Você costuma verificar se os apelidos da aplicação de desenvolvimento e produção
	estão iguais?\newline
	- Nunca verificou.
	\item Realiza uma navegação por todas as páginas da aplicação?\newline
	- Sim.
	\item Verifica se todas as tabelas foram criadas?\newline
	- Sim, verifica todas as tabelas.
\end{enumerate}

%Fim das respostas do Desenvolvedor 8

\end{apendicesenv}
