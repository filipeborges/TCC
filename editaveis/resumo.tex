\begin{resumo}

Nos últimos anos tem se tornado cada vez mais comum o desenvolvimento de aplicações por usuários finais, seja através do desenvolvimento de planilhas robustas, de \textit{queries} em banco de dados, e até mesmo de aplicações mais simples. Esta modalidade de desenvolvimento é nomeada como \textit{End User Development}, EUD. Dentro da administração pública federal brasileira, um órgão tem reconhecido estes esforços dos usuários finais e adotou um modelo de desenvolvimento para aplicações feitas sob essa modalidade, o que tem gerado um alto relato de satisfação interna. Apesar disso, a flexibilidade no desenvolvimento destas aplicações pode gerar uma negligência a padrões e boas práticas, o que gera oportunidades de melhoria na forma de desenvolvimento das mesmas. O propósito deste trabalho é propor uma solução de apoio ao desenvolvimento baseado em EUD dentro do órgão em questão, contribuindo para a melhoria na qualidade e manutenibilidade das aplicações. A pesquisa é classificada como qualitativa, aplicada e exploratória, com o uso dos procedimentos bibliográfico, documental e pesquisa-ação. Ao fim, espera-se verificar a eficiência da solução de apoio aplicando-a em um projeto piloto.


 \vspace{\onelineskip}
    
 \noindent
 \textbf{Palavras-chaves}: latex. abntex. editoração de texto.
\end{resumo}
