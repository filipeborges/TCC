\begin{resumo}

Nos últimos anos tem se tornado cada vez mais comum o desenvolvimento de aplicações por usuários finais, seja pelo do desenvolvimento de planilhas robustas, de \textit{queries} em banco de dados, e até mesmo de aplicações a partir de ferramentas especializadas. Esta modalidade de desenvolvimento é denominada na literatura como \textit{End User Development} (EUD). Na administração pública federal brasileira, um órgão tem reconhecido estes esforços dos usuários finais e adotou um modelo de desenvolvimento para aplicações feitas sob essa modalidade, resultando em um alto índice de satisfação interna. Uma das características específicas do EUD é a flexibilidade no desenvolvimento destas aplicações, o que pode levar os desenvolvedores a negligenciarem padrões e boas práticas de desenvolvimento de software. Diante deste contexto, o objetivo deste trabalho foi propor uma solução (sistema) de apoio ao desenvolvimento baseado em EUD no órgão em questão, contribuindo para a melhoria na qualidade e manutenibilidade das aplicações. A pesquisa é classificada como qualitativa, aplicada e explicativa. Como resultado houve a construção do sistema de apoio ao desenvolvimento baseado em EUD para o órgão em questão. Com o trabalho foi possível verificar que há uma necessidade de apoio ao desenvolvedor EUD dentro do órgão, visto a variedade na experiência dos mesmos e nos padrões e técnicas usados pelos diferentes departamentos da instituição.




 \vspace{\onelineskip}
    
 \noindent
 \textbf{Palavras-chaves}: \textit{end user development}. desenvolvimento descentralizado. apex. desenvolvimento por usuário final. \textit{application express}.
\end{resumo}
