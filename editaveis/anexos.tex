\begin{anexosenv}

\partanexos

\chapter{Perguntas da entrevista}

\textbf{Nome do entrevistado (Opcional):}\newline\newline
\textbf{Descrição do estagiário}\newline\newline
Unidade:\newline    
    1- Qual é o curso que você faz?\newline
    2- Quantos semestres são o curso?\newline
    3- Está em qual semestre do curso?\newline
    4- Qual é o nome da instituição de ensino?\newline
    5- Quanto tempo de estágio?\newline
\newline
\textbf{Papel do estagiário na organização.}\newline\newline
1- Como você enxerga o seu papel na instituição? (A posição dentro da instituição, as responsabilidades, e etc…)\newline
    2- Você considera a sua contribuição relevante para a instituição?\newline
    \newline
\textbf{Efetividade do curso}\newline\newline
    1- Você acha que o curso o preparou bem para as tarefas que realizou?\newline
    2- Qual a sua opinião com relação a efetividade do curso?\newline
    3- Como você avalia a qualidade das aplicações em que deu manutenção?\newline
\newline
\textbf{Boas práticas e padrões}\newline\newline
    1- Você adota algum padrão ou alguma boa prática no desenvolvimento/manutenção das aplicações?\newline
    2- Teria alguma sugestão de boa prática/padrão que realiza que poderia melhorar de alguma forma a qualidade das aplicações desenvolvidas?\newline
    \newline\newline\newline
\textbf{Processo de Desenvolvimento}\newline\newline\newline
\textbf{Requisitos}\newline\newline
    1- Como você faz para levantar os requisitos do sistema que foi solicitado por alguma unidade?\newline
    2- Como você aborda o cliente para levantar os requisitos do sistema?\newline
    3- Como você armazena ou documenta o que foi passado pelo cliente (requisitos)? Existe algum meio de armazenar eles diretamente no sistema?\newline
    4- Você costuma fazer reunião com o cliente em outras etapas do desenvolvimento, para levantar outros requisitos e/ou esclarecer os que já existem?\newline
    \newline
\textbf{Design}\newline\newline
    1- Como você elabora o modelo de dados do sistema (ferramentas, MER, ML)? Se nunca elaborou, considera-o importante?\newline
    2- Você valida tecnicamente este modelo com área responsável ou com algum padrão disponível (nomenclatura e formas normais)? Se nunca validou, considera-o importante?\newline
\newline
\textbf{Codificação}\newline\newline
    1- Você costuma utilizar código PL/SQL, HTML e Javascript em mais da metade da aplicação?\newline
    2- Você procurar escrever os seus códigos PL/SQL, SQL, HTML e Javascript de forma legível?\newline
    3- Você segue algum padrão de codificação ou boas práticas (constantes, nomes significativos) para PL/SQL, SQL, HTML e Javascript?\newline
    4- Você costuma usar algum editor de código ou o próprio APEX, para escrever seus códigos?\newline
    5- Você costuma depurar o código?\newline
    6- Você costuma classificar os erros encontrados (falha de execução ou comportamento inesperado), de forma que possa ajudá-lo na abordagem de resolução?\newline
    7- Você costuma comentar o código que faz?\newline
\newline
\textbf{Teste}\newline\newline
    1- Como você testa as suas aplicações?\newline
    2- Você costuma testar todas as páginas de aplicação que desenvolve?\newline
    3- Você costuma armazenar os resultados dos testes que realiza?\newline
    4- Você costuma a elaborar casos de testes para sua aplicação?\newline
    5- Seria bom testes automatizados de interface para sua aplicação (selenium)?\newline
\newline
\textbf{Implantação}\newline\newline
    1- Você costuma migrar as definições dos objetos de banco (tabelas, sequences, views e etc) junto com a aplicação?\newline
    2- Você costuma verificar se os apelidos da aplicação de desenvolvimento e produção estão iguais?
    3- Realiza uma navegação por todas as páginas da aplicação?\newline
    4- Verifica se todas as tabelas foram criadas?\newline
\newline
\chapter{Segundo Anexo}

Texto do segundo anexo.

\end{anexosenv}

