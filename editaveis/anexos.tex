\begin{anexosenv}

\partanexos
\chapter{Perguntas da entrevista}

\begin{flushleft}
\textbf{Descrição do desenvolvedor}
\end{flushleft}
Objetivo(s):
\begin{itemize}
\item Obter informações sobre o perfil dos desenvolvedores.
\begin{itemize}
\item Código derivado: Perfil do Desenvolvedor;
\end{itemize}
\end{itemize}
Unidade de alocação:\newline\newline    
    1- Qual é o curso que você faz?\newline
    2- Quantos semestres são o curso?\newline
    3- Está em qual semestre do curso?\newline
    4- Qual é o nome da instituição de ensino?\newline
    5- Quanto tempo de estágio?\newline
\newline
\textbf{Papel do desenvolvedor na organização.}\newline\newline
Objetivo(s):
\begin{itemize}
\item Obter informações sobre o papel do desenvolvedor no órgão.
\begin{itemize}
\item Código derivado: Papel do Desenvolvedor;
\end{itemize}
\end{itemize}
6- Como você enxerga o seu papel na instituição? (A posição dentro da instituição, as responsabilidades, e etc…)\newline
    7- Você considera a sua contribuição relevante para a instituição?\newline
    \newline
\textbf{Efetividade do curso}\newline\newline
Objetivo(s):
\begin{itemize}
\item Obter informações sobre a preparação dos novos desenvolvedores.
\begin{itemize}
\item Código derivado: Preparação;
\end{itemize}
\end{itemize}
    8- Você acha que o curso o preparou bem para as tarefas que realizou?\newline
    9- Qual a sua opinião com relação a efetividade do curso?\newline
    10- Como você avalia a qualidade das aplicações em que deu manutenção?\newline
\newline
\textbf{Boas práticas e padrões}\newline\newline
Objetivo(s):
\begin{itemize}
\item Obter informações sobre as práticas realizadas no desenvolvimento.
\begin{itemize}
\item Código derivado: Estilo e Design de codificação;
\end{itemize}
\end{itemize}
	11- Você adota algum padrão ou alguma boa prática no desenvolvimento/manutenção das aplicações?\newline
    12- Teria alguma sugestão de boa prática/padrão que realiza que poderia melhorar de alguma forma a qualidade das aplicações desenvolvidas?\newline
    \newline
\textbf{Requisitos}\newline\newline
Objetivo(s):
\begin{itemize}
\item Obter informações sobre como se dá o levantamento de requisitos nos departamentos da instituição.
\begin{itemize}
\item Código derivado: Elicitação de Requisitos;
\end{itemize}
\item Obter informações de como é armazenado e gerenciado os requisitos do sistema a ser desenvolvido.
\begin{itemize}
\item Código derivado: Gerenciamento de Requisitos;
\end{itemize}
\item Obter informações de como é englobado as mudanças de requisitos ao sistema.
\begin{itemize}
\item Código derivado: Gerenciamento de Requisitos;
\end{itemize}
\end{itemize}
    13- Como você faz para levantar os requisitos do sistema que foi solicitado por alguma unidade?\newline
    14- Como você aborda o cliente para levantar os requisitos do sistema?\newline
    15- Como você armazena ou documenta o que foi passado pelo cliente (requisitos)? Existe algum meio de armazenar eles diretamente no sistema?\newline
    16- Você costuma fazer reunião com o cliente em outras etapas do desenvolvimento, para levantar outros requisitos e/ou esclarecer os que já existem?\newline\newline\newline\newline
\textbf{Design}\newline\newline
Objetivo(s):
\begin{itemize}
\item Obter informações sobre como é realizado a modelagem do sistema.
\begin{itemize}
\item Código derivado: Modelagem do sistema;
\end{itemize}
\end{itemize}
    17- Como você elabora o modelo de dados do sistema (ferramentas, MER, ML)? Se nunca elaborou, considera-o importante?\newline
    18- Você valida tecnicamente este modelo com área responsável ou com algum padrão disponível (nomenclatura e formas normais)? Se nunca validou, considera-o importante?\newline
\newline
\textbf{Codificação}\newline\newline
Objetivo(s):
\begin{itemize}
\item Obter informações sobre a depuração de erros do sistema.
\begin{itemize}
\item Código derivado: Depuração;
\end{itemize}
\item Obter informações sobre estilos de codificação.
\begin{itemize}
\item Código derivado: Estilo e \textit{Design} de codificação;
\end{itemize}
\item Obter informações sobre o ambiente de codificação.
\begin{itemize}
\item Código derivado: Ambiente de codificação;
\end{itemize}
\end{itemize}
    19- Você costuma utilizar código PL/SQL, HTML e Javascript em mais da metade da aplicação?\newline
    20- Você procurar escrever os seus códigos PL/SQL, SQL, HTML e Javascript de forma legível?\newline
    21- Você segue algum padrão de codificação ou boas práticas (constantes, nomes significativos) para PL/SQL, SQL, HTML e Javascript?\newline
    22- Você costuma usar algum editor de código ou o próprio APEX, para escrever seus códigos?\newline
    23- Você costuma depurar o código?\newline
    24- Você costuma classificar os erros encontrados (falha de execução ou comportamento inesperado), de forma que possa ajudá-lo na abordagem de resolução?\newline
    25- Você costuma comentar o código que faz?\newline
\newline\newline
\textbf{Teste}\newline\newline
Objetivo(s):
\begin{itemize}
\item Obter informações sobre a realização dos testes, bem como a forma de realizá-los.
\begin{itemize}
\item Código derivado: Teste;
\end{itemize}
\end{itemize}
    26- Como você testa as suas aplicações?\newline
    27- Você costuma testar todas as páginas de aplicação que desenvolve?\newline
    28- Você costuma armazenar os resultados dos testes que realiza?\newline
    29- Você costuma a elaborar casos de testes para sua aplicação?\newline
    30- Seria bom testes automatizados de interface para sua aplicação (selenium)?\newline
\newline
\textbf{Implantação}\newline\newline
Objetivo(s):
\begin{itemize}
\item Obter informações sobre a migração, para a produção, do sistema.
\begin{itemize}
\item Código derivado: Implantação;
\end{itemize}
\end{itemize}
    31- Você costuma migrar as definições dos objetos de banco (tabelas, sequences, views e etc) junto com a aplicação?\newline
    32- Você costuma verificar se os apelidos da aplicação de desenvolvimento e produção estão iguais?\newline
    33- Realiza uma navegação por todas as páginas da aplicação?\newline
    34- Verifica se todas as tabelas foram criadas?\newline
\newline
\chapter{Segundo Anexo}

Texto do segundo anexo.

\end{anexosenv}

